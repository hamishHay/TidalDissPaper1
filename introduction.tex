\newpage
\section{Introduction}

\hbadness=10000
\vbadness=10000

Thermal energy in the interiors of outer Solar System icy satellites is supplied primarily by radiogenic decay and tidal dissipation. Radiogenic decay plays a role in large icy satellites with a significant portion of silicate material in their interiors \citep{hussmann2006subsurface}. This role, however, diminishes with decreasing mass of silicate material. Small satellites have a high surface area to volume ratio, and consequently thermal energy generated from radioactive decay is lost on a timescale much less than the age of the Solar System. Yet, several small and medium sized icy satellites have confirmed global oceans, suggesting greater interior heating than that provided by radiogenic decay alone.

%Europa exhibits an induced magnetic field consistent with a global layer of salty, liquid water beneath its surface, as measured by the Galileo spacecraft \citep{zimmer2000subsurface, kivelson2000galileo, hand2007empirical}. This is also the case with Ganymede and Callisto, although these liquid oceans are likely to exist at least a few hundred kilometers beneath their surface \citep{zimmer2000subsurface, kivelson2000galileo}. 

This work focuses on Titan and Enceladus. Several interior models and lines of evidence suggest Titan contains a subsurface ocean \citep{sohl2003interior, bills2011rotational, iess2012tides, baland2014titan, mitri2014shape, sohl2014structural}. Enceladus also shows strong evidence of a liquid ocean beneath its surface. Originally it was thought that this liquid reservoir was localised beneath the South Polar Terrain (SPT) of the satellite \citep[e.g.,][]{collins2007enceladus}. However, recent modelling of the degree-2 gravity field was consistent with a global ocean with greatest thickness beneath the SPT, although such models are non-unique \citep{iess2014gravity,mckinnon2015effect}. Most recently, the large forced libration of Enceladus indicates a decoupling of the icy shell from its interior, and thus the presence of an ocean of global extent \citep{thomas2016enceladus}.  

%Tidal dissipation is currently considered the most promising mechanism to provide energy to heat these oceans over geological timescales, and shall be the main focus of this paper.

\textbf{This paper is intended to introduce and verify a new numerical model for solving thin shell ocean dynamics in planetary bodies. The model is therefore ideal for investigating ocean dissipation in a variety of icy satellites. Assumptions made in the development of the numerical code reflect those made in the semi-analytical models to which our results are compared. This ensures the most accurate validation of the numerical model. We also introduce bottom drag into the model, an extension that is only possible numerically and in simplified scaling analysis \citep{chen2013tidal}. We explore this drag regime for oceans on Titan and Enceladus.}

\subsection{Tidal Dissipation}

Any satellite that passes through a varying gravitational potential will experience some form of tidal dissipation. The time varying gravitational potential may be a result of the satellite's orbital eccentricity and/or obliquity, as well as any non-synchronous rotation. For a satellite in (near) synchronous rotation, the gravitational tidal potential will vary periodically over the satellite's orbit. The changing potential does mechanical work on the satellite, and a portion of this work is converted to thermal energy. This process is known as tidal dissipation. As long as sufficient orbital or rotational energy remains in the system (in the form of eccentricity, obliquity or non-synchronous rotation), tidal dissipation will occur.

Both the solid and fluid regions of a satellite will experience tidal dissipation. Despite this, and the overwhelming evidence for and abundance of subsurface oceans in the icy satellites, the majority of dissipation studies have focused on only solid-body tides, \citep[e.g.,][]{moore2000tidal, tobie2005tidal,roberts2008tidal, beuthe2013spatial}.
Most terrestrial tidal dissipation occurs within the oceans, and while Earth has a complex dynamic between tidally-induced ocean flow and its continents, it illustrates the importance of considering ocean dissipation.

The effect of ocean dissipation in outer planet satellites was first considered for Titan by \citet{sagan1982tide}, who analytically derived expressions to estimate dissipated energy in a global hydrocarbon surface ocean. In doing so they attempted to explain Titan's relatively high eccentricity. The same problem was then approached numerically by \citet{sears1995tidal}, who solved the Laplace Tidal Equations (LTEs) to derive time averaged estimates of dissipated energy within a hydrocarbon ocean of varying thicknesses. Sears' numerical model is the basis for the model described in this paper.

More recently, \citet{tyler2008strong,tyler2009ocean,tyler2011tidal,tyler2014comparative} has done extensive work on ocean dissipation, showing that thermal energy released by an ocean can theoretically prevent a subsurface liquid from freezing. By exploring how ocean thickness and drag coefficient effect dissipation, \citet{tyler2011tidal} discovered ocean dissipation resonances. These resonances tend to occur for particular ocean thicknesses, where oceanic planetary waves resonantly interact with the periodic tidal forcing, allowing enhanced tidal flow and consequently enhanced tidal dissipation. \citet{matsuyama2014tidal} developed a similar model to that used by \citet{tyler2011tidal}, adding the effects of ocean loading, self-attraction, and deformation of the solid regions. These effects were shown to alter the position and magnitude of these dissipative resonances. \citet{kamata2015tidal} has also modelled the effects of an icy shell on Love number resonances, but did not include any ocean dynamics in their approach.

\citet{tyler2011tidal}, \citet{matsuyama2014tidal}, and \citep{chen2013tidal} all considered Rayleigh (linear) drag in their models.  \citet{chen2013tidal} also developed a set of scaling laws to model ocean dissipation in the bottom (quadratic) drag regime. The numerical model presented in this work is capable of solving the LTEs using both Rayleigh and bottom drag, as is typical in terrestrial ocean dissipation studies \citep{taylor1920tidal,jeffreys1921tidal,zahel1977global,egbert2001estimates,jayne2001parameterizing}. \citet{sears1995tidal} used both Rayleigh and bottom drag simultaneously in his model. In contrast, we adopt a bimodal approach to illustrate the differences between Rayleigh and bottom drag. These two regimes are briefly described below.

\subsection{Rayleigh Drag}

A linear formulation of drag was first introduced as the Guldberg-Mohn approximation of \textit{virtual} internal friction in 1876 \citep{neumann1968ocean}. Now known as Rayleigh drag, the approximation describes drag within a fluid that is proportional and opposite to the fluid's velocity. That is, the drag force per unit mass $\bm{F_d} = -\alpha \bm{u}$, where $\alpha$ is some drag time scale known as the coefficient of Rayleigh drag with units of $\si{\per\second}$. The flow velocity is $\bm{u}$. 

Rayleigh drag can be though of as a macroscopic description of drag between adjacent fluid elements in a moving liquid.

\subsection{Bottom Drag}

Terrestrial ocean dissipation studies often employ a drag model that scales with the square of the fluid's velocity. This quadratic dependence of drag on flow velocity is often referred to as \textit{bottom drag} \citep{gill1982atmosphere}, and arises due to turbulent flow interacting with some bottom boundary, such as the ocean floor. Large tangential shear stresses associated with this interface generate a turbulent boundary layer where there is a significant transfer of momentum from the flow. While such turbulent flow cannot be resolved at the scale of planetary ocean simulations, the bottom drag coefficient $c_D$ is empirically derived to include the frictional effect of this turbulence at the planetary scale.

In this work we investigate the effects of bottom drag on ocean dissipation for both Titan and Enceladus, structuring the paper as follows. Firstly, we introduce our numerical method in section \ref{sec:method}, describing the grid structure and numerical solver as well as its current limitations. Secondly, we then apply this model to both Titan (section \ref{sec:results_Titan}) and Enceladus (section \ref{sec:results_Enceladus}) in turn for each drag model, examining how dissipation differs between each case. The Rayleigh drag results are compared to semi-analytical solutions of \citet{matsuyama2014tidal}. We also compare the bottom drag results against scaling laws developed by \citep{chen2013tidal}.






