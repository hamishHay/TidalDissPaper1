\newpage
\section{Introduction}

\hbadness=10000
\vbadness=10000

The primary source of heat generation in the outer Solar System icy satellites is energy lost via tidal dissipation. A result of this heating is the formation and potentially long term stable existence of global subsurface oceans, as has already been confirmed in several of the icy satellites. Europa has been shown to have an induced magnetic field consistent with a global layer of salty, liquid water beneath its surface \citep{zimmer2000subsurface, kivelson2000galileo, hand2007empirical}. This is also the case with Ganymede and Callisto, although these liquid oceans are likely to exist at least a few hundred kilometers beneath the surface \citep{zimmer2000subsurface, kivelson2000galileo}. Saturn's largest moon, Titan, has been interpreted to have a conductive liquid layer beneath its surface due a Schumann-like resonance \citep{beghin2010titan}, and this is supported by measurements of its mean moment of inertia \citep{bills2011rotational}. Enceladus also shows signs of a liquid ocean confined to its southern hemisphere. Evidence for this includes emission and composition of cryovolcanic plumes from a series of tidally induced shear faults over its south pole \citep{hansen2011composition}, as well as a detected negative mass anomaly consistent with a subsurface ocean \citep{iess2014gravity}. 

Despite this overwhelming evidence and abundance of subsurface oceans in the icy satellites, previous tidal dissipation studies have not been able to account for the energy needed to sustain these oceans over time. Many of these studies have primarily focused on tidal dissipation in the solid regions of the satellite \citep[e.g.,][]{moore2000tidal, roberts2008tidal, beuthe2013spatial}, ignoring the effects of a subsurface ocean. This is somewhat surprising as ocean dissipation is expected to dominate energy loss in the system, as is the case for Earth.

Other previous works on tidal dissipation within the icy satellites have considered a rather idealised case with only a global surface ocean, completely ignoring the effects of a solid lid \citep{sagan1982tide, sears1995tidal, sohl1995tidal, tyler2008strong, matsuyama2014tidal}. By assuming that there is no solid lid, much of this work is not directly applicable to the icy satellites of the outer Solar System as they only have subsurface oceans. The presence of a solid lid is likely to dampen tidal dissipation in the liquid ocean, and thus it is important to examine the potential consequences that solid lids may have on tidal heating.

Tidal dissipation not only plays a role in the thermal evolution of the icy satellites, but but it also affects their rotational and orbital state. By studying and considering the effects of dissipation in the solid and liquid regions of the outer Solar System icy satellites, we hope to not only understand and help constrain their current thermal state and interior structure, but also how the satellite's interior, rotation and orbit has evolved over time. 