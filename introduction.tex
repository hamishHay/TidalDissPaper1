\newpage
\section{Introduction}

\hbadness=10000
\vbadness=10000

Thermal energy in the interiors of outer Solar System icy satellites is supplied primarily by radiogenic decay and tidal dissipation. Radiogenic decay plays a significant role in large icy satellites with bulk densities that indicate a significant portion of the satellite is silicate \citep{hussmann2006subsurface}. This role, however, diminishes with decreasing mass of silicate material. Small satellites have a high surface area to volume ratio, and consequently thermal energy generated from radioactive decay is lost on a timescale much less than the age of the Solar System. Yet, several icy satellites have confirmed to global oceans, suggesting greater interior heating than that provided by radiogenic decay alone.

Europa exhibits an induced magnetic field consistent with a global layer of salty, liquid water beneath its surface, as measured by the Galileo spacecraft \citep{zimmer2000subsurface, kivelson2000galileo, hand2007empirical}. This is also the case with Ganymede and Callisto, although these liquid oceans are likely to exist at least a few hundred kilometers beneath their surface \citep{zimmer2000subsurface, kivelson2000galileo}. 

The detection of a Schumann-like resonance at Saturn's largest moon, Titan, has been interpreted to suggest a conductive liquid layer beneath its surface \citep{beghin2010titan}. This is measurements of Titan's mean moment of inertia \citep{bills2011rotational}. Enceladus also shows signs of containing a liquid ocean. Original evidence suggested that this liquid reservoir was localised to the south polar regions of the satellite \citep[e.g.,][]{collins2007enceladus}. However, recent analysis of the degree 2 gravity field suggests a global ocean with greatest depth at the south polar region, although such models are non-unique \citep{iess2014gravity,mckinnon2015effect}. Most recently, the large forced libration of Enceladus indicates a decoupling of the icy shell from its interior, and thus the presence of an ocean of global extent \citep{thomas2015enceladus}.  

Tidal dissipation is currently considered the most promising mechanism that provides the energy to support the existence of these oceans over geological time, and shall be the main focus of this paper.

\subsection{Tidal Dissipation}

Any satellite that passes through a varying gravitational potential will experience some form of tidal dissipation. The varying gravitational potential may be a result of the satellite's orbital eccentricity and/or obliquity, as well as any non-synchronous rotation. For a satellite in (near) synchronous rotation, the gravitational tidal potential will vary periodically with the satellite's orbit. The changing potential does mechanical work on the satellite, and a portion of this is converted to thermal energy, or heat. This is process is known as tidal dissipation. As long as sufficient orbital or rotational energy remains in the system (in the form of eccentricity, obliquity or non-synchronous rotation), tidal dissipation will occur.

Both the solid and fluid parts of an orbiting satellite will experience tidal dissipation. 
However, despite the overwhelming evidence for and abundance of subsurface oceans in the icy satellites, the majority of dissipation studies have focused on solid-body tides, \citep[e.g.,][]{moore2000tidal, tobie2005tidal,roberts2008tidal, beuthe2013spatial}.
Most terrestrial tidal dissipation occurs within the oceans, and while Earth has a complex dynamic between tidally-induced ocean flow and its continents, it illustrates the importance of considering ocean dissipation.

The significance of ocean dissipation in outer planet satellites was first recognised following Voyager measurements of the saturation and surface temperature of Titan. The surface temperature, which is such that methane and ethane exist as stable liquids on its surface, prompted much speculation over the existence of a hydrocarbon ocean. \citet{sagan1982tide} tackled this problem analytically, deriving expressions to estimate dissipated energy in a hydrocarbon surface ocean on Titan. In doing so they attempted to explain Titan's relatively high eccentricity. The same problem was then approached numerically by \citet{sears1995tidal}, who solved the shallow water equations to derive orbitally time averaged estimates of dissipated energy within a hydrocarbon ocean of varying depths. Sears' numerical model is the basis for the model described in this paper.

More recently, \citet{tyler2008strong,tyler2009ocean,tyler2011tidal,tyler2014comparative} has done extensive work on tidal dissipation, showing that thermal energy released by ocean dissipation can theoretically prevent a subsurface liquid from freezing. By exploring how varying ocean depth and friction coefficient effect dissipation, \citet{tyler2011tidal} discovered ocean dissipation resonances. These resonances tend to occur at particular ocean depths, where the geometry of the ocean massively enhances tidal flow and consequently tidal dissipation. \citet{matsuyama2014tidal} developed a similar model to that used by \citet{tyler2011tidal}, adding the effects of ocean loading, self-attraction, and deformation of the solid regions. These effects alter the position and magnitude of dissipative resonances. \citet{kamata2015tidal} also modelled the effects of an icy shell on Love number resonances, but did not investigate any ocean dissipation.

Both \citet{tyler2011tidal} and \citet{matsuyama2014tidal} used Rayleigh friction in their models. The numerical model presented in this work will instead solve the shallow water equations using bottom (quadratic) friction, as is typically the case with terrestrial ocean dissipation studies \citep{taylor1920tidal,jeffreys1921tidal,zahel1977global,egbert2001estimates,jayne2001parameterizing}. \citet{sears1995tidal} used both Rayleigh and bottom friction simultaneously in his model. In contrast, we adopt a more bimodal approach to illustrate the differences between Rayleigh and bottom friction. These two friction models are briefly described below.

\subsection{Rayleigh Friction}

A linear formulation of friction was first introduced as the Guldberg-Mohn approximation of \textit{virtual} internal friction in 1876 \citep{neumann1968ocean}. Now known as Rayleigh friction, the approximation describes drag within a fluid that is proportional and opposite to the fluid's velocity. That is, the drag force $\bm{F_d} = -\alpha \bm{u}$, where $\alpha$ is some drag time scale known as the coefficient of Rayleigh friction with units of $\si{\per\second}$. The flow velocity is $\bm{u}$. When possible, $\alpha$ is derived empirically.

Rayleigh friction can be though of as a macroscopic description of drag between adjacent fluid elements in a moving liquid.

\subsection{Bottom Friction}

Ocean dissipation studies often employ friction that scales with the square of the fluid's velocity. This quadratic dependence of drag on flow velocity is often referred to as \textit{bottom friction} \citep{gill1982atmosphere}, and arises due to turbulent flow interacting with some bottom boundary, such as the ocean floor. Large tangential shear stresses associated with this interface generate a turbulent boundary layer where there is a significant transfer of momentum from the flow. While such turbulent flow cannot be resolved at the scale of planetary ocean simulations, the bottom friction coefficient $c_D$ is empirically derived to include the frictional effect of this turbulence at the planetary scale.

In this work we investigate the effects of bottom friction on ocean dissipation for both Titan and Enceladus, structuring the paper as follows. Firstly, we compare the numerical model against the semi-analytical solutions of \citet{matsuyama2014tidal} for a global surface ocean on Titan, minus the effects of ocean loading, self-attraction, and deformation of the solid regions. Secondly, we explore the differences between Rayleigh friction and bottom friction by comparing numerical model results for each friction model, for a global surface ocean on Titan. We then repeat this analysis for a global surface ocean on Enceladus. Finally, we compare our results against scaling laws developed by \citep{chen2013tidal}.




