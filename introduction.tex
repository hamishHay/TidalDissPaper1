\newpage
\section{Introduction}

\hbadness=10000
\vbadness=10000

The primary sources of thermal energy in the interiors of the outer Solar System icy satellites are radiogenic decay and tidal dissipation. Radiogenic decay plays a significant role in heating the largest icy satellites (ref), but this role diminishes with decreasing mass of silicate material. Small satellites, those with a high surface area to volume ratio, lose their heat quickly. Thermal energy generated from radioactive decay will be rapidly lost, cooling down the satellite's interior. Yet, several icy satellites have been confirmed to harbour global and potentially non-global subsurface oceans, suggesting significant interior heating.

Europa exhibits an induced magnetic field consistent with a global layer of salty, liquid water beneath its surface, as measured by the Galileo spacecraft \citep{zimmer2000subsurface, kivelson2000galileo, hand2007empirical}. This is also the case with Ganymede and Callisto, although these liquid oceans are likely to exist at least a few hundred kilometers beneath their surface \citep{zimmer2000subsurface, kivelson2000galileo}. 

The detection of a Schumann-like resonance at Saturn's largest moon, Titan, has been interpreted to suggest a conductive liquid layer beneath its surface \citep{beghin2010titan}. This is supported by measurements of Titan's mean moment of inertia \citep{bills2011rotational}. Enceladus also shows signs of a liquid ocean confined to its southern hemisphere. Evidence for this includes the emission and composition of cryovolcanic plumes from a series of tidally induced shear faults over its south pole \citep{hansen2011composition}, as well as the detection of a negative mass anomaly consistent with a subsurface ocean or sea \citep{iess2014gravity}. 

Tidal dissipation is currently considered the most promising mechanism that supports the long-term existence of these oceans over, and shall be the main focus of this paper.

\subsection{Tidal Dissipation}

Any satellite that passes through a varying gravitational potential will experience some form of tidal dissipation. The varying gravitational potential may be a result of the satellite's orbital eccentricity and/or obliquity, as well as any non-synchronous rotation. For a satellite in (near) synchronous rotation, the gravitational tidal potential will vary periodically with the satellite's orbit. The changing potential acts on the satellite by doing mechanical work, which in turn is converted to thermal energy, or heat. This is process is known as tidal dissipation. As long as sufficient orbital or rotational energy remain in the system (in the form of eccentricity, obliquity or non-synchronous rotation), tidal dissipation will occur.

Both the solid and fluid parts of an orbiting satellite will experience tidal dissipation. However, despite the overwhelming evidence for and abundance of subsurface oceans in the icy satellites, the majority of dissipation studies have focused on solid-body tides, \citep[e.g.,][]{moore2000tidal, roberts2008tidal, beuthe2013spatial}. The majority of terrestrial tidal dissipation occurs within the oceans, and while Earth has a complex dynamic between tidally-induced ocean flow and the continents, it illustrates the importance of considering ocean dissipation.

The importance of ocean dissipation for Solar System satellites other than our own was first recognised after Voyager measurements of the saturation and surface temperature of Titan {ref}. \citet{sagan1982tide} tackled this problem in an analytical fashion, deriving expressions to estimate dissipated energy in a hydrocarbon surface ocean on Titan. In doing so they attempted to explain Titan's high eccentricity.

The same problem was then approached numerically by \citet{sears1995tidal}, solving the shallow water equations to derive orbitally time averaged estimates of dissipated energy within a hydrocarbon of varying depths. Sears' numerical model is the basis for the model described in this paper. 

More recently, \citet{tyler2008strong} etc... 
