\section{Methodology}

This section describes some of the theory and methods involved in this work. The governing equations of the numerical model presented here are discussed in section \ref{subsec:LTE}. Following this, a description of the equations discretisation and the numerical scheme required to solve them are outlined in section \ref{subsec:model}.

\subsection{Laplace Tidal Equations \label{subsec:LTE}}

The equations of motion and mass conservation that describe ocean flow in the shallow water limit are called the Laplace Tidal Equations (LTE) (also known as the shallow water equations) \citep{lamb1932hydrodynamics}. The main assumption in their derivation is that radial (vertical) ocean flow is negligible when compared to lateral flow. This is indeed a good approximation for global oceans, where lateral flow spans a much greater distance than the depth of the ocean. The conservation of mass (eq. \ref{eq:mass}) and momentum (eq. \ref{eq:mom}) that make up the LTE, including both Rayleigh and bottom friction, are given as \citep{sears1995tidal,tyler2008strong,matsuyama2014tidal}:

%Adds vertical space between equations
%\setlength{\jot}{8pt}
%environment centres all equations

\vspace{-0.5cm}
\begin{gather}
\partial_t \eta + \nabla \cdot \left(h \bm{u}\right) = 0\, , \label{eq:mass}\\
\begin{aligned} 
\partial_t \bm{u} + 2 \bm{\Omega} \times \bm{u} + \alpha\bm{u} + \frac{c_D}{h} \left|\bm{u}\right| \bm{u}  + g \nabla \eta \\ = (1 + k_2 - h_2) \nabla U_2 \, . \label{eq:mom}\\
\end{aligned} 
\end{gather}

Equation \ref{eq:mass} consists of two terms. The first is the time rate of change of vertical sea surface displacement about some equilibrium level. $\eta$ denotes this displacement. The second term represents the divergence of the surface velocity vector, $\bm{u} \equiv (u, v)$, where $u$ and $v$ are the eastward and northward velocity components respectively. In our calculations, we assume the ocean's undisturbed depth, $h$, to be constant. 

The term on the right hand side of Equation \ref{eq:mom} is an applied force per unit mass. $\nabla U_2$ is the gradient of the degree-2 tidal potential (discussed in section ...). It is multiplied by Love's reduction factor involving the degree-2 tidal Love numbers, $k_2$ and $h_2$. Love's first number, $k_2$, is a proportionality constant accounting for the additional tidal potential due to the elastic redistribution of mass on the satellite. $h_2$, the second Love number, accounts for the tidal potential arising from solid body surface displacement of the satellite \citep{love1911some}.

The time derivative of velocity is given by the first term on the left hand side of the momentum equation. It is balanced by four other acceleration per unit mass terms on the left hand side. The first of these terms is the coriolis acceleration per unit mass, where $\Omega$ is the satellite's rotational angular velocity. Rayleigh and bottom friction are described in the next two terms, where $\alpha$ and $c_D$ are the Rayliegh (linear)and bottom (quadratic) friction coefficients respectively \citep{sears1995tidal,chen2013tidal}. The last term on the right hand side of Equation \ref{eq:mom} is the gravitational restoring acceleration per unit mass. It acts to balance changes in sea surface displacement, given by the gradient of equilibrium displacement, $\nabla \eta$ .

\subsection{Numerical Model \label{subsec:model}}

The numerical model outlined below is based on the models extensively discussed in \citet{zahel1973diurnalk,zahel1978influence} and \citet{sears1994tidal,sears1995tidal}.

\subsubsection{Finite Difference Expansion}

As in \citep{sears1995tidal}, we expand equations \ref{eq:mass} and \ref{eq:mom} in a semi-implicit finite difference scheme in spherical coordinates. By expanding $\bm{u}$ into its components, the momentum equation becomes,

\vspace{-0.6cm}
\begin{multline}
u_{ij}^{t+1} =  \left[ \,2 \Omega \bar{v}_{ij} \sin{\lambda_i} \vphantom{\frac{c_D}{h}\sqrt{\left(u_{ij}^{t}\right)^2}} - \alpha u_{ij}^{t} \right. \\ 
- \frac{c_D}{h}\sqrt{\left(u_{ij}^{t}\right)^2 + \left(\bar{v}_{ij}^{t}\right)^2}\cdot\left(u_{ij}^{t}\right)^2 - \frac{g}{R \cos{\lambda_i}} \frac{\partial \eta_{ij}^{t}}{\partial \phi_j} \\  
+ \left.\left(1 + k_2 - h_2\right) \frac{1}{R \cos{\lambda_i}} \frac{\partial U_{2,ij}^{t}}{\partial \phi_j} \right]  \Delta t + u_{ij}^{t} \, , \label{eq:momu_fd}
\end{multline}
\vspace{-0.6cm}
\begin{multline}
v_{ij}^{t+1} =  \left[ \,2 \Omega \bar{u}_{ij} \sin{\lambda_i} \vphantom{\frac{c_D}{h}\sqrt{\left(u_{ij}^{t}\right)^2}} - \alpha v_{ij}^{t} \right. \\ 
- \frac{c_D}{h}\sqrt{\left(\bar{u}_{ij}^{t}\right)^2 + \left(v_{ij}^{t}\right)^2}\cdot\left(v_{ij}^{t}\right)^2 - \frac{g}{R} \frac{\partial \eta_{ij}^{t}}{\partial \lambda_i} \\  
+ \left.\left(1 + k_2 - h_2\right) \frac{1}{R} \frac{\partial U_{2,ij}^{t}}{\partial \lambda_i} \right]  \Delta t + v_{ij}^{t} \, , \label{eq:momv_fd}
\end{multline}

whereas the mass equation becomes, 

\begin{equation}
\eta_{ij}^{t+1} = 
-\frac{h}{R \cos{\lambda_i}}\left(
\frac{\partial \left(v_{ij}^{t+1} \cos{\lambda_i}\right)}{\partial	\lambda_i}  
+\frac{\partial u_{ij}^{t+1}}{\partial	\phi_j}\right)
\Delta t
+ \eta_{ij}^{t}\, . \label{eq:mass_fd}
\end{equation}




