\section{Ocean Dissipation in Enceladus \label{sec:results_Enceladus}}

\begin{figure*}[!t]
    \centering
    \begin{subfigure}[t]{0.9\linewidth} % contains the two plots in a single figure
        \includegraphics[width=\linewidth]{Figures/enceladus_linear}
        \phantomcaption
        \label{fig:lincEccEncel}
    \end{subfigure}
    \begin{subfigure}[t]{0\linewidth} % the hidden unwanted image
         \includegraphics[width=\linewidth]{Figures/enceladus_linear}
         \phantomcaption
         \label{fig:linObliqEncel} 
    \end{subfigure}
    \vspace{-0.5cm}
\caption{Numerical average surface ocean dissipation for Enceladus under the eccentricity (left) and obliquity (right) tides. The logarithm of dissipated energy is shown as function of ocean depth, $h$, and Rayleigh friction coefficient, $\alpha$. All simulations were performed with \SIrange{1}{3}{\degree} grid resolution. \label{fig:linEncel}}
\end{figure*}

\subsection{Rayleigh Friction}

As with our Titan results, we calculated globally averaged surface heat flux for over 3000 simulations in the $h$-$\alpha$ parameter space. These results are shown for the eccentricity and obliquity tides in Figure \ref{fig:linEncel}.

There are several more gravity wave resonances found for Enceladus than Titan in both tidal components. The eccentricity tide excites resonances at \SIlist{1.3;1.9;3.8;8.7;29;50;360}{m}; a total of 7 resonances, whereas Titan only has 3 (Figure \ref{fig:lincEccTitan}). By solving the LTEs using both the eccentricity-radial and libration tides simultaneously, we capture any coupling between these two tidal components and find no new resonances. Increasing the resolution of the parameter space would like reveal more resonances, but this is only likely for oceans with $h_0 <$ \SI{10}{\metre}. 

The most dissipative eccentricity resonance is also the deepest, with an average surface heat flux of \SI{4.6}{\watt\per\square\metre} at $\alpha\sim$ \SI{2e-6}{\per\second}, three orders of magnitude greater than Titan's largest resonance. This is equivalent to a total power output of $\sim$ \SI{3610}{\giga\watt}, well in excess of the observed value by two to three orders of magnitude \citep{spencer2006cassini}. 

The stark colour differences between the eccentricity and obliquity tide plots in Figure \ref{fig:linEncel} is a result of Enceladus' negligible obliquity ($\theta_0 =$ \SI{0.00014}{\degree}, \citep{chen2013tidal}). This causes average surface dissipation to range from \SIrange{e-7}{e-14}{\watt\per\square\metre} over our explored parameter space, which has an almost negligible effect on the thermal and orbital evolution of the satellite. Gravity wave resonances are found at \SIlist{1.6;2.7;5.5;18;160}{\metre}, with the characteristic Rossby wave resonance extending diagonally across the parameter space from around $h=$ \SI{500}{\metre}.

Away from shallow oceans, we once again see excellent agreement between the numerical and analytical results, in terms of both the resonant ocean thickness and magnitude of the dissipation. Much of the parameter space has a discrepancy of $<$ \SIrange{1}{5}{\percent}, with this increasing to $\sim$ \SI{10}{\percent} for resonances. As demonstrated in Figure \ref{fig:conv}, this can easily be decreased with higher resolution simulations, at the expense of computational run time.


\subsection{Bottom Friction}

\subsection{Comparison with Scaling Laws}

\subsection{Implications for Enceladus}