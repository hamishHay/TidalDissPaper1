\section{Conclusions}

We have designed and implemented a numerical model based on \citet{sears1995tidal} to solve the dissipative Laplace Tidal Equations on a sphere. This model, valid only in the shallow water limit and assuming a global surface ocean, has been used to model ocean flow and its associated tidal dissipation over a range of ocean thickness and drag coefficients for both Titan and Enceladus. We neglect the effects of ocean loading and self-attraction.

Modelling is performed with both Rayleigh (linear) and bottom (quadratic) drag models. The former represents internal drag between two adjacent fluid parcels \citep{neumann1968ocean}, while the latter is a global scale approach to incorporating the effects of a macro-scale turbulent boundary layer at some solid-fluid interface \citep{gill1982atmosphere}.

Rayleigh drag results were compared to that of \citet{matsuyama2014tidal} yielding mostly excellent agreement over much of the explored parameter space, providing important validation to our numerical model. The presence and position of ocean dissipative resonances were replicated well in both Rayleigh and bottom drag simulations. Importantly, we have also demonstrated good agreement between our bottom drag numerical results and the scaling laws of \citet{chen2013tidal} away from gravity wave resonances, providing further validation of the bottom drag implementation in our model.

For Titan, we found that Rossby wave resonance associated with the obliquity tide becomes independent of ocean depth away from shallow ($h \sim\SI{1}{\kilo\metre}$) oceans. This is also shown in the \citet{chen2013tidal} obliquity tide scaling law. Such depth independence allows us to place loose constraints on the bottom drag coefficient in Titan's ocean by modelling semimajor axis evolution under the assumption of a deep ocean \citep{sohl2014structural,baland2014titan}. We find that bottom drag coefficients of less that \SI{e-4} or greater than \SI{2e-2} are acceptable to recreate Titan's present day semimajor axis, assuming a starting semimajor axis equal to Hyperion's (the next outer satellite of Saturn). Both limits are an order of magnitude away from the canonical Earth value ($\sim 0.002$ \citep{egbert2001estimates}), indicating that Titan has either anomalously low or high drag in its ocean. We currently cannot distinguish between these two regimes. 

Enceladus eccentricity tide ocean resonances are found to be extremely dissipative, as in \citet{tyler2011tidal, matsuyama2014tidal}. However, the ocean depths associated with the deepest of these resonances are too shallow to be a probable scenario on Enceladus. We hope to model tidal flow for a spatially varying ocean thickness on Enceladus in the future, as this will likely effect resonant thicknesses and may lead to more localised heating at the SPT. Additionally, it is important to incorporate the effects of an icy shell atop these oceans to understand how this limits ocean dynamics and dissipation, which will be explored in future work. 