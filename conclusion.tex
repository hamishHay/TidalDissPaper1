\section{Conclusions}

We have designed and implemented a numerical model based on \citet{sears1995tidal} to solve the dissipative Laplace Tidal Equations on a sphere. This model, valid only in the shallow water limit and assuming a global surface ocean, has been used to model ocean flow and its associated tidal dissipation over a range of ocean thicknesses and drag coefficients for both Titan and Enceladus. We neglect the effects of ocean loading and self-attraction.

Modelling is performed with both Rayleigh (linear) and bottom (quadratic) drag models. The former represents internal drag between two adjacent fluid parcels \citep{neumann1968ocean}, while the latter is a global scale approach to incorporating the effects of a macro-scale turbulent boundary layer at some solid-fluid interface \citep{gill1982atmosphere}.

Rayleigh drag results were compared to that of \citet{matsuyama2014tidal} yielding mostly excellent agreement over much of the explored parameter space, providing important validation to our numerical model. Ocean dissipative resonances were replicated well in both Rayleigh and bottom drag simulations. Importantly, we have also demonstrated good agreement between our bottom drag numerical results and the scaling laws of \citet{chen2013tidal} away from gravity-wave resonances, providing further validation of the bottom drag implementation in our model.

For Titan, all gravity-wave resonances occur in very thin oceans and as such  are unlikely to exist. The exception is the Rossby-wave resonance associated with the obliquity tide, which becomes independent of ocean thickness away from thin oceans ($h_0 \leqslant\SI{1}{\kilo\metre}$) in the bottom drag regime. This is also shown in the \citet{chen2013tidal} obliquity tide scaling law. Such a feature means that for a thick ocean on Titan, which is thought to be the case \citep{sohl2014structural}, then ocean dissipation becomes dependent on only bottom drag coefficient. For an Earth like bottom drag coefficient we find that ocean dissipation induced by Titan's obliquity tide can reduce its rate of outward orbital migration by  $\sim\SI{40}{\percent}$. Additionally, measurement of $da/dt$ could place constraints on the bottom drag coefficient in Titan's ocean because this resonance becomes independent of ocean thickness. 

Enceladus' eccentricity tide ocean resonances are found to be extremely dissipative, as in \citet{tyler2011tidal, matsuyama2014tidal}. However, the ocean thicknesses where these resonances form are almost certainly too small to be present on Enceladus. We hope to model tidal flow for a spatially varying ocean thickness on Enceladus in the future, as this will likely effect resonant thicknesses and may lead to more localised heating at the SPT. Additionally, it is important to incorporate a solid icy lid into our model to understand how this affects ocean dynamics and dissipation, which will be explored in future work. 