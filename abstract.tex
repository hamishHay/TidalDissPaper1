Icy satellites that contain subsurface oceans require sufficient thermal energy to prevent their liquid interiors from freezing. We develop a numerical model to solve the Laplace Tidal Equations on a sphere to model tidal flow and thermal energy dissipation in these oceans, neglecting the presence of an icy lid. The model is applied to Titan and Enceladus, where we explore how Rayleigh (linear) and bottom (quadratic) drag terms affect dissipation. The latter drag regime can only be applied numerically. We find excellent agreement between our results and recent analytical work. Obliquity tide Rossby wave resonant features become independent of ocean depth under the bottom drag regime for thick oceans. This depth independent feature allows us to constrain bottom drag coefficients of less than \num{e-4} or greater than \num{2e-2} for Titan's subsurface ocean, based on its past-projected semimajor axis evolution. Eccentricity tides on Enceladus can provide significant ocean dissipation in resonant ocean thicknesses, yet are unlikely to do so given the shallow position of these resonances in our examined parameter space.