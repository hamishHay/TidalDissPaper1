Icy satellites that contain subsurface oceans require sufficient thermal energy to prevent the liquid portion of their interiors from freezing. We develop a numerical model to solve the Laplace Tidal Equations on a sphere in order to model tidal flow and thermal energy dissipation in these oceans, neglecting the presence of an icy lid. The model is applied to Titan and Enceladus, where we explore how Rayleigh (linear) and bottom (quadratic) drag terms affect dissipation. The latter drag regime can only be applied numerically. We find excellent agreement between our results and recent analytical work. Obliquity tide Rossby wave resonant features become independent of ocean depth under the bottom drag regime for thick oceans. We show that for Titan, dissipation associated with this Rossby wave resonance acts to dampen the rate of outward orbital migration. Gravity wave resonances can act to cause inward migration, although this is unlikely due to the thin oceans required to form such resonances. Eccentricity tides on Enceladus can provide significant ocean dissipation at resonant gravity-wave resonances, but are also unlikely to do so as thin oceans are required to form these resonances.