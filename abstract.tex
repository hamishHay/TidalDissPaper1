Icy satellites that contain subsurface oceans require sufficient thermal energy to prevent the liquid portion of their interiors from freezing. We develop a numerical finite difference model to solve the Laplace Tidal Equations on a sphere in order to simulate tidal flow and thermal energy dissipation in these oceans, neglecting the presence of an icy lid. The model is applied to Titan and Enceladus, where we explore how Rayleigh (linear) and bottom (quadratic) drag terms affect dissipation. The latter drag regime can only be applied numerically. We find excellent agreement between our results and recent analytical work. Obliquity tide Rossby-wave resonant features become independent of ocean thickness under the bottom drag regime for thick oceans. We show that for Titan, dissipation from this Rossby-wave resonance can act to dampen the rate of outward orbital migration by up to \SI{40}{\percent} for Earth-like values of bottom drag coefficient. Gravity-wave resonances can act to cause inward migration, although this is unlikely due to the thin oceans required to form such resonances. The same is true of all eccentricity tide resonances on Enceladus, such that dissipation becomes negligible for thick oceans under the bottom drag regime.  